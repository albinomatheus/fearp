% ------------------------------------------------------------------------
% ------------------------------------------------------------------------
% Modelo de Trabalho Acadêmico da FEA-RP (tese de doutorado, dissertação de
% mestrado e trabalhos monográficos em geral) em conformidade com 
% ABNT NBR 14724:2011 e Diretrizes para Confecção de Teses e Dissertações da USP
% ------------------------------------------------------------------------
% ---------------------------------------------------------
\documentclass[rec]{fearp}
\autor          {Matheus Alves Albino}
\titulo         {Modelo para Monografias, Dissertações e Teses em LaTeX}
\data           {2017} 
\local          {Ribeirão Preto}
\preambulo      {Template em \LaTeX\ desenvolvido no \imprimirdepartamento \ da \imprimirfaculdade.}
\orientador[Orientador/Orientadora]{Prof./Prof.ª}{X X}
\engtitle{LaTeX Template for Theses, Dissertations and Monographs} % Título em inglês do trabalho

% ----------------------------------------------------------------------------%
%         INFORMAÇÕES DO PDF                                                  %
% ----------------------------------------------------------------------------%
\hypersetup{
  pdftitle       = {\imprimirtitulo},
  pdfsubject     = {\imprimirpreambulo},
  pdfproducer    = {\LaTeX},
  pdfcreator     = {pdflatex},
  colorlinks     = true,
  linkcolor      = black,            % Cor de links internos
  citecolor      = black,            % Cor dos links para a bibliografia
  filecolor      = black,            % Cor dos links
  urlcolor       = blue,
  bookmarksdepth = 4}
% Palavras-chave para a ficha catalográfica
\palavraschave{LaTeX}{ABNT}{AbnTeX2}{FEARP}
% ----------------------------------------------------------------------------%
%         BIBLIOGRAFIA/CITAÇÕES                                               %
% ----------------------------------------------------------------------------
% ABNT
\usepackage[bibjustif, abnt-full-initials=yes, abnt-repeated-title-omit=yes, abnt-emphasize=it, alf]{abntex2cite} 
%------------------------------------
% Estilo APA
%\usepackage[natbibapa]{apacite} 
%\renewcommand{\BBAB}{e} % Troca "and" por "e" em apacite
%-----------------------------------
% Estilo Harvard
% \usepackage{harvard} % para definições de bibliografia harvard (Usa com "and")
% \citationstyle{dcu} 
% \renewcommand{\harvardand}{\&} % troca "e" por "\&"
%------------------------------------
% Estilo de Vancouver
%\bibliographystyle{vancouver}
%\setcitestyle{numbers}
% ----------------------------------------------------------------------------%
%         ELEMENTOS PRÉ-TEXTUAIS                                              %
% ----------------------------------------------------------------------------%
\begin{document}
\imprimircapa
\newpage
\begin{center}
\vspace*{18cm}
Prof. Dr. Marco Antônio Zago \\
Reitor da \imprimiruniversidade \\
\vspace{20pt}
Prof. Dr. Dante Pinheiro Martinelli  \\
Diretor da \imprimirfaculdade \\
\vspace{20pt}
Prof. Dr. Renato Leite Marcondes \\
Chefe do \imprimirdepartamento
\end{center} % Verso da capa
\imprimirfolhaderosto

% Ficha catalográfica de acordo com o modelo utilizado pela Biblioteca Central de Ribeirão Preto
\begin{fichacatalografica}
\newpage
\thispagestyle{empty}
{
\normalsize Autorizo a reprodução e divulgação total ou parcial deste trabalho, por qualquer meio convencional ou eletrônico, para fins de estudo e pesquisa, desde que citada a fonte.
\normalsize
\vfill
\begin{center}
\begin{tabular}{c}
  {\scriptsize \MakeUppercase{}}
\end{tabular}
\vspace{-0.2cm}
\begin{tabular}{|m{0.2cm}p{11.6cm}m{0.2cm}|} \hline
  \hspace{0.3cm} & & \\
  & \ABNTautordatabib & \\
  \hspace{0.2cm}  & \hspace{0.3cm} \imprimirtitulo \ / \imprimirautor \ -- \imprimirlocal, \imprimirdata. & \\
  & \hspace{0.65cm} \pageref{LastPage}f.: il.; 30 cm & \\
  & \hspace{0.4cm} & \\
  & \hspace{0.6cm} \imprimirpreambulo \ --\ \imprimiruniversidade & \\
  & \hspace{0.6cm} \imprimirorientadorRotulo : \ABNTorientadordatabib & \\
  & & \\
  & \hspace{0.6cm} \imprimirchaves. & \\
  & & \\
  & \hspace{4.75cm} & \\
  \hline
\end{tabular}
\end{center}
}
\vspace{2.9cm}
\end{fichacatalografica}   % Ficha Catalográfica
\begin{folhadeaprovacao}
  \begin{center}
    {\normalsize \MakeUppercase{\imprimirautor}}
        \vspace*{2.5cm}
    \begin{center}
      \normalsize \textbf{\imprimirtitulo}
    \end{center}
    \vspace*{2.5cm}
    \hspace{.45\textwidth}
    \begin{minipage}{.5\textwidth}
        \imprimirpreambulo\\
        \par
        Área de Concentração: 
    \end{minipage}%
  \end{center}
  \vspace*{2cm}
\centering
  \textbf{Data de Aprovação:}\\
  \par
  \_\_\_\_\_/\_\_\_\_\_/\_\_\_\_\_\_\_\_\_ \\
  \vspace{1.5cm}
  \textbf{Banca Examinadora:}
  \assinatura{\textbf{\titulacaoorientador \ \imprimirorientador} \\ \imprimirorientadorRotulo} 
  \assinatura{\textbf{Professor} \\ Avaliador 1}
  \assinatura{\textbf{Professor} \\ Avaliador 2}
\end{folhadeaprovacao} % Folha de Aprovação

\begin{dedicatoria}
\begin{flushright}
\vspace*{\fill}
\begin{minipage}{.52\textwidth}
A dedicatória é um elemento pré-textual opcional.
\end{minipage}
\end{flushright}
\end{dedicatoria}
\begin{agradecimentos}
Os agradecimentos são um elemento pré-textual opcional.
\end{agradecimentos}

\begin{epigrafe}
\begin{flushright}
\vspace*{\fill}
\begin{minipage}{.52\textwidth}
A epígrafe é um elemento pré-textual opcional.
\end{minipage}
\end{flushright}
\end{epigrafe}

% Resumo
\begin{resumo}
\tipotrabalho{Manual}
\vspace{\onelineskip}

O resumo em língua vernácula é um elemento pré-textual obrigatório, normalmente entre 100 e 300 palavras que deve indicar os tópicos essenciais do trabalho científico. É recomendável que esse texto inclua a contextualização do objeto de estudo, o objetivo geral do trabalho (isto é, sua questão de investigação), uma linha tratando do referencial teórico, a hipótese central a ser testada, a metodologia e as principais conclusões. É recomendado que o resumo seja acompanhado de 3 a 6 palavras-chave, que descrevam o trabalho. Para elaborar as palavras-chave, o autor pode utilizar as que constam do Vocabulário Controlado do SiBi/USP: \url{http://143.107.154.62/Vocab/Sibix652.dll/}. O Código JEL para textos em Economia pode ser consultado em: \url{https://pt.wikipedia.org/wiki/C\%C3\%B3digos\_de\_classifica\%C3\%A7\%C3\%A3o\_JEL}.

\vspace{\onelineskip}

\textbf{Palavras-chave}: 

\jel{Y20} 
\end{resumo}

% Abstract
\begin{resumo}[Abstract]
\engtype{Manual} % Tipo de trabalho em inglês
\vspace{\onelineskip}

O resumo em inglês é um elemento pré-textual obrigatório. 

\vspace*{\onelineskip}
\noindent
\textbf{Keywords}: \\
\jel{} 
\end{resumo}

% Lista de ilustrações 
% \newpage
% \pdfbookmark[0]{\listfigurename}{lof}
% \listoffigures*
% \cleardoublepage

% Lista de tabelas 
% \pdfbookmark[0]{\listtablename}{lot}
% \listoftables*
% \cleardoublepage

% Lista de quadros (opcional)
% \pdfbookmark[0]{\listtablename}{loq}
% \listofquadros*
% \cleardoublepage

% Lista de abreviaturas e siglas (opcional)
% \begin{siglas}
%   \item
% \end{siglas}

% Sumário 
\phantom{x}
\pdfbookmark[0]{\contentsname}{toc}
\tableofcontents*
\cleardoublepage

\mainmatter 
\pagestyle{meuestilo}

% ----------------------------------------------------------------------------%
%         ELEMENTOS TEXTUAIS                                              %
% ----------------------------------------------------------------------------%
 
\chapter*{Introdução}
\addcontentsline{toc}{chapter}{INTRODUÇÃO} % Para capítulos sem numeração que aparecem no sumário

Uma breve introdução, geralmente sem subdivisão em partes e que deve conter um breve panorama ou contextualização do problema a ser discutido, uma revisão bibliográfica com os trabalhos mais significativos da área e a indicação da questão de investigação com duas ou mais variáveis que guiam a pesquisa. 

Outros elementos recomendados para a introdução são a metodologia utilizada e as fontes, os objetivos gerais e específicos e a explicação sobre o desenvolvimento do trabalho, isto é, o conteúdo de cada capítulo.


\chapter{Opções da classe}

A classe traz customizações do abnTeX2 \url{http://abnTeX2.googlecode.com} para a Faculdade de Economia, Administração e Contabilidade de Ribeirão Preto (FEA-RP/USP). Fornece customização de capa, folha de rosto, formatação do texto e dos títulos e definição de macros auxiliares. 

Para trabalhar com \LaTeX\ e com a classe abn\TeX\ , são sugeridos os seguintes \textit{links}:
\begin{itemize}
  \item Distribuição \LaTeX\ para windows: \url{https://www.tug.org/texlive/}
  \item Editor de \LaTeX\ gratuito: \url{http://texstudio.sourceforge.net/}
  \item JabRef -- Gerenciador de arquivos \texttt{.bib}: \url{http://jabref.sourceforge.net/}
  \item Mendeley -- Gerenciador de artigos e bibliografia: \url{http://www.mendeley.com/}
  \item Informações da classe Abn\TeX : \citeonline{abntex22013}
  \item Editor de Textos Sublime Text: \url{http://www.sublimetext.com/}
  \item SumatraPDF -- Leitor de PDF Gratuito amigável para trabalhar com \LaTeX\ no Sublime Text: \url{http://jabref.sourceforge.net/}
  \item Para tópicos específicos sobre citações e referência no formato ABNT: \cite{abntex22013b} e \cite{abntex22013c}
  \item Diretrizes para apresentação de dissertações e teses da USP: documento eletrônico e impresso (ABNT): \url{http://www.teses.usp.br/index.php?option=com_content&view=article&id=52&Itemid=67}
  \item Para o Modelo Canônico do AbnTeX2, ver \citeonline{abntex22013d}
  \item Para converter arquivos em \LaTeX\ para \texttt{.rtf}: \url{http://www.sciweavers.org/l2rtf}
\end{itemize}

\section{Customizações de \texttt{abntex2}}

A opção default imprime a versão final do texto, com as informações do autor, nome do orientador e elementos pretextuais. Todas as opções de classe são repassadas para Abn\TeX 2.
\begin{verbatim}
    \documentclass{fearp}
\end{verbatim} 

Há também opções predefinidas de capa e folha de rosto:
\begin{table}[htb]
\IBGEtab{
\caption{Opções predefinidas de capa e folha de rosto}
\label{predef}
}{%
\begin{tabular}{ll}
\toprule
Opção & Curso/Departamento \\
\midrule
\texttt{rec} & Departamento de Economia da FEA-RP/USP \\
\texttt{rcc} & Departamento de Contabilidade da FEA-RP/USP \\
\texttt{rad} & Departamento de Administração da FEA-RP/USP \\
\texttt{ppgao} & Pós-Graduação em Administração de Organizações \\
\texttt{ppge} & Pós-Graduação em Economia \\
\texttt{ppgcc} &  Pós-Graduação em Controladoria e Contabilidade\\
\texttt{eae} & Departamento de Economia da FEA-USP \\
\texttt{ead} & Departamento de Administração da FEA-USP \\
\texttt{eac} & Departamento de Contabilidade e Atuária da FEA-USP \\
\bottomrule
\end{tabular}
}{
\nota{Os comandos \textbackslash universidade\{\} , \textbackslash faculdade\{\} e \textbackslash departamento\{\} devem ser especificados no preâmbulo no caso de não utilizar as predefinições de capa.}
}
\end{table}

\section{Opção \texttt{blind}}

A opção \texttt{blind} imprime versões da capa e da folha de rosto sem as informações do nome do autor e orientador. Ao utilizar essa opção, também são omitidos os ambientes \texttt{agradecimentos} e \texttt{dedicatoria}. No caso do Departamento de Economia, deve-se retirar do Resumo e do texto o código JEL. 
\begin{verbatim}
    \documentclass[blind, rec]{fearp}
\end{verbatim}

\section{Soluções para citações e bibliografia}

Além do estilo de citação ABNT, as Diretrizes USP permitem a utilização dos sistemas de referências e estilos de bibliografia da APA (\emph{American Psychological Association}), do Estilo de Vancouver e da norma ISO 690:2010 -- \emph{Information and documentation: guidelines for bibliographic references and citations to information resource}.

\subsection{APA}
Para selecionar o estilo APA, basta adicionar ao preâmbulo:
\begin{verbatim}
\usepackage[natbibapa]{apacite} 
\renewcommand{\BBAB}{e} % Troca "and" por "e" em apacite
\end{verbatim}
E mudar o estilo de bibliografia para: 
\begin{verbatim}
\bibliographystyle{apacite} % APA
\end{verbatim}

As Diretrizes USP a serem seguidas estão disponíveis em: \url{http://dx.doi.org/10.11606/9788573140576}. Ver também a Documentação do pacote \texttt{apacite} \url{http://ctan.math.utah.edu/ctan/tex-archive/biblio/bibtex/contrib/apacite/apacite.pdf#page=10}

Uma outra alternativa é utilizar o estilo Harvard aliado ao parâmetro \texttt{dcu}:

\begin{verbatim}
\usepackage{harvard} 
\citationstyle{dcu} 
\renewcommand{\harvardand}{\&} % troca "e" por "\&"
\end{verbatim}

\subsection{Vancouver}
Para selecionar o estilo de Vancouver, basta adicionar ao preâmbulo: 
\begin{verbatim}
\bibliographystyle{vancouver}
\setcitestyle{numbers}
\end{verbatim}
E mudar o estilo de bibliografia para:
\begin{verbatim}
\bibliographystyle{vancouver} % Vancouver
\end{verbatim}

Para as normas ISO, as Diretrizes USP estão disponíveis em: \url{http://dx.doi.org/10.11606/9788573140569}.
Ver também a documentação de \texttt{Vancouver.bst}: \url{http://get-software.net/biblio/bibtex/contrib/vancouver/vancouver.pdf}.

\subsection{ISO}

Para as normas ISO, as Diretrizes USP a serem seguidas estão disponíveis em: \url{http://dx.doi.org/10.11606/9788573140590}.


% ----------------------------------------------------------------------------%
%         ELEMENTOS PÓS-TEXTUAIS                                              %
% ----------------------------------------------------------------------------%
% \bibliographystyle{apacite} % APA
% \bibliographystyle{vancouver} % Vancouver
% \bibliographystyle{dcu} % Harvard
\bibliographystyle{abntex2-alf}
\bibliography{referencias/referencias.bib}
\begin{anexosenv}
\chapter{Exemplo de Anexo}
Um anexo.
\begin{figure}[H]
 \caption{Brasão da Universidade de São Paulo}
    \centering
    \includegraphics[width=.3\textwidth]{figuras/brasao_usp.eps}
    \fonte{Universidade de São Paulo}
    \label{fig:brasao}
\end{figure}

\ctable[caption = {Título da figura}, 
label = tab:label, 
figure, 
pos = h
]{c}{\tnote[]{Nota: Uma figura utilizando o pacote ctable.}}
{\includegraphics[width=.3\textwidth]{figuras/brasao_usp.eps}}

\end{anexosenv}
\begin{apendicesenv}
\chapter{Exemplo de Apêndice}
Um exemplo de apêndice.
\begin{quadro}[htb]
\IBGEtab{
\caption{Exemplo de quadro}
\label{predef}
}{%
\begin{tabular}{l}
\toprule
Departamentos \\
\midrule
Departamento de Economia da FEA-RP/USP \\
Departamento de Contabilidade da FEA-RP/USP \\
Departamento de Administração da FEA-RP/USP \\
Departamento de Economia da FEA-USP \\
Departamento de Administração da FEA-USP \\
Departamento de Contabilidade e Atuária da FEA-USP \\
\bottomrule
\end{tabular}
}{
}
\end{quadro}
Um exemplo de tabelas usando o pacote \texttt{ctable}:
\ctable[
caption = {Título da tabela}, 
label = tab:label, 
nosuper,
notespar,
doinside = \footnotesize, 
pos = h
]{ll}{
\tnote[]{Nota: Uma tabela utilizando o pacote ctable.}}
{ 
\toprule
Opção & Curso/Departamento \\
\midrule
\texttt{rec} & Departamento de Economia da FEA-RP/USP \\
\texttt{rcc} & Departamento de Contabilidade da FEA-RP/USP \\
\texttt{rad} & Departamento de Administração da FEA-RP/USP \\
\texttt{ppgao} & Pós-Graduação em Administração de Organizações \\
\texttt{ppge} & Pós-Graduação em Economia \\
\texttt{ppgcc} &  Pós-Graduação em Controladoria e Contabilidade\\
\texttt{eae} & Departamento de Economia da FEA-USP \\
\texttt{ead} & Departamento de Administração da FEA-USP \\
\texttt{eac} & Departamento de Contabilidade e Atuária da FEA-USP \\
\bottomrule
}
\end{apendicesenv}
\end{document}

